
\begin{summary}

In the Philippines, traditional manual pothole detection methods have proven ineffective and error-prone, hindering efficient road maintenance. This study introduces the Eyeway system, an AIoT prototype tailored for real-time pothole detection in the Philippine context. By integrating lightweight YOLO architectures (YOLOv9-tiny, YOLOv10-nano, and YOLOv11-nano), the system aims to improve detection accuracy and computational efficiency. Previous studies utilizing limited datasets established initial feasibility, but this research employs an expanded dataset of 16,054 images, encompassing diverse environmental conditions and road surface characteristics. The comparative analysis establishes new benchmarks for pothole detection in resource-constrained environments, demonstrating the viability of nano/tiny models for practical applications. The Eyeway system, designed specifically for the Philippines, integrates computer vision algorithms, AI-on-edge technology, and map visualization to detect potholes in real-time. This study evaluates the effectiveness of the Eyeway system in enhancing road maintenance practices, bridging the technological gap, and improving the efficiency of road infrastructure management in the Philippines.

\end{summary}